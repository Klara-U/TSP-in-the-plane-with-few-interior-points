\documentclass[a4paper, 11pt]{article}
\usepackage[slovene]{babel}
\usepackage[utf8]{inputenc}
\usepackage[T1]{fontenc}
\usepackage{amsfonts,amsmath,amssymb}
\usepackage{amsthm}
\usepackage{amsmath}
\usepackage{amssymb}

\newtheorem*{teorem}{Teorem}


\begin{document}

\begin{titlepage}
    \begin{center}
        \Large
        Univerza v Ljubljani\\
        Fakulteta za matematiko in fiziko\\
        Finančna matematika\\

        \vspace*{6cm}
        
        \Huge
        \textbf{TSP in the plane with a few interior points}

        \vspace*{3cm}

        \Large
        11.naloga pri predmetu\\
        Finančni praktikum

        \vspace*{5cm}

        \Large
        Nikodin Sedlarevič in Klara Uršič\\
        Ljubljana, 2022
    \end{center}
\end{titlepage}

\section*{Problem trgovskega potnika}

Problem trgovskega potnika je v kombinatorični optimizaciji zelo znan NP-težek problem, kar pomeni, da še ne poznamo algoritma,
s katerim bi ga uspeli rešiti v polinomskem času. Danih imamo $n$ mest $\{1,2,\dots,n\}$, za vsak par $i$ in $j$ izmed 
le-teh poznamo razdaljo $d(i,j)$ med njima. Cilj je ugotoviti najkrajšo pot, ki jo trgovski potnik opravi, če obišče čisto vsa 
dana mesta natanko enkrat, na koncu pa se vrne v izhodiščno mesto.

\section*{Konveksni Evklidski problem trgovskega potnika}

Konveksni Evklidski problem trgovskega potnika je različica problema trgovskega potnika, pri katerem imamo danih $P$ točk v Evklidski 
ravnini, razdalje med točkami pa so Evklidske. Nadalje postavimo še pogoj, da morajo točke ležati v konveksnem položaju. Če je 
izpolnjena ta zahteva, postane naš problem veliko lažje rešljiv, saj najkrajšo pot med vsemi točkami namreč tvori kar rob lupine, ki jo
tvorijo konveksno postavljene točke.

\section*{Naloga}

V naši nalogi bomo opazovali, kaj se zgodi, ko končno število točkam iz Konveksnega Evklidskega problema trgovskega potnika, dodamo 
notranjo točko, ki leži znotraj konveksne lupine, in kaj se zgodi, ko dodamo dve, tri, \dots To variacijo se da za končno število točk 
v ravnini rešiti z dinamičnim programiranjem, mi pa bomo poleg tega proučili še, kako na rešitev problema vpliva število in postavitev 
točk. Kaj se zgodi, ko število notranjih točk povečamo oz. zmanjšamo, kakšen je najslabši položaj notranjih točk, če točke na robu 
tvorijo pravilni poligon, kaj se zgodi, če zunanje točke ležijo na robu, niso pa ekvidistančne\dots ?

\section*{Reševanje}

Da bomo lahko ustrezno preučili vse možne scenarije, bomo v izbranem programskem jeziku s pomočjo dinamičnega programiranja napisali 
kodo za rešitev tega problema, ki bo zadoščala naslednjemu teoremu.

\begin{teorem}
    Poseben primer Evklidskega problema trgovskega potnika z le nekaj notranjimi točkami je možno rešiti v naslednji časovni in prostorski 
    zahtevnosti. Tu $n$ predstavlja število vseh točk in $k$ število točk, ki ležijo v notranjosti konveksne lupine. $(1)$ V času $O(k!kn)$
    in prostoru $O(k)$. $(2)$ V času $O(2^kk^2n)$ in prostoru $O(2^kkn)$.
\end{teorem}

Znotraj kode bomo nato spreminjali različne parametre, da bomo ustvarili želene situacije, ki jih bomo želeli preučiti, ter izrisali grafe, 
ki bodo še grafično prikazali dobljene rezultate.


\end{document}